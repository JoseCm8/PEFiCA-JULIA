\documentclass{article} 
\usepackage{tikz} 
\begin{document} 
\tikzstyle{isometric}=[x={(0.710cm,-0.410cm)},y={(0cm,0.820cm)},z={(-0.710cm,-0.410cm)}] 
\tikzstyle{dimetric}=[x={(0.9cm,-0.1cm)},z={(0cm,0.9cm)},y={(+0.3cm,+0.3cm)}] 
\tikzstyle{vista}=[x={(0.9cm,-0.3cm)},z={(0cm,0.9cm)},y={(+0.75cm,+0.3cm)}] 
\begin{center} 
\begin{tikzpicture}[vista] 
% estilos de linea y relleno 
\tikzstyle{carga} = [ultra thick,latex-]  
% definicion apoyo empotrado 
% #1: ubicación del apoyo 
% #2: escala 
 \newcommand{\apoyoemz}[2]{ 
\coordinate (O) at #1; 
\begin{scope}[scale=#2] 
\draw[ultra thin,fill=black!60] (O) ++ (-0.25,-0.25,0) -- ++(0.50,0,0) -- ++(0,0.50,0) -- ++(-0.50,0,0) -- cycle; 
\end{scope} 
 } 
% definicion apoyo de segundo genero 
% #1: ubicación del apoyo 
% #2: angulo de rotacion 
 \newcommand{\apoyoseg}[2]{ 
\coordinate (O) at #1; 
\begin{scope}[rotate around={#2:(O)}] 
\fill [black!10](O) ++(-0.45,-0.433013) rectangle ++(0.9,-0.2); 
\draw [thick] (O) ++(-0.45,-0.433013) -- ++(0.9,0); 
\draw (O) -- ++(-0.25,-0.433013) -- ++(0.45,0) -- cycle; 
\end{scope} 
 } 
% definicion apoyo de primer genero 
% #1: ubicación del apoyo 
% #2: angulo de rotacion 
 \newcommand{\apoyopri}[2]{ 
\coordinate (O) at #1; 
\begin{scope}[rotate around={#2:(O)}] 
\fill [black!10](O) ++(-0.45,-0.433013) rectangle ++(0.9,-0.2); 
\draw [thick] (O) ++(-0.45,-0.433013) -- ++(0.9,0); 
\draw (O) ++(0,-0.216506) circle (0.216506); 
\end{scope} 
 } 
% definicion carga puntual 
% #1: ubicación de la carga 
% #2: rótulo de la carga 
% #3=1: carga positiva, #3=-1: carga negativa 
% #4=1,#5=0: carga en x, #4=0,#5=1: carga en y 
% #6=0: carga entrando al nudo, #6=1: carga saliendo del nudo 
% #7: ubicación del rótulo de la carga 
\newcommand{\cargapun}[7]{ 
\path #1 ++(#6*#4,#6*#5) coordinate (O); 
\draw [carga] (O) -- ++(-1.0*#3*#4,-1.0*#3*#5) 
node [#7] {#2}; 
 } 
% definición de carga distribuida 
% #1: ubicación del nudo inicial del elemento 
% #2: ubicación del nudo final del elemento 
% #3: componentes de un vector en y-local de norma 0.5 
% #4: etiqueta de la carga 
% #5: ubicación de la etiqueta de la carga 
\newcommand{\cargadis}[5]{ 
\coordinate (I) at #1; 
\coordinate (F) at #2; 
\coordinate (N) at #3; 
\draw[fill=orange!20,opacity=0.5] (I)++(N)--(I)--(F)--++(N)--cycle 
node[#5,opacity=1.0] {#4}; 
  } 
% cota horizontal 
% #1: coordenada punto inicial () 
% #2: coordenada punto final () 
% #3: rotulo de la cota 
% #4: separación entre puntos y cota 
% #5: separación entre puntos y inicio de marcas 
% #6: separación entre cota y fin de marcas 
\newcommand{\cotahori}[6]{ 
\path #1 ++(0,#4) coordinate (A); 
\path #2 ++(0,#4) coordinate (B); 
\draw [latex-latex] (A) -- (B) 
node [midway,fill=white] {#3}; 
\path #1 ++(0,#5) coordinate (C); 
\path #1 ++(0,#4+#6) coordinate (D); 
\draw (C) -- (D); 
\path #2 ++(0,#5) coordinate (C); 
\path #2 ++(0,#4+#6) coordinate (D); 
\draw (C) -- (D); 
 } 
% cota vertical 
% #1: coordenada punto inicial () 
% #2: coordenada punto final () 
% #3: rotulo de la cota 
% #4: separación entre puntos y cota 
% #5: separación entre puntos y inicio de marcas 
% #6: separación entre cota y fin de marcas 
\newcommand{\cotavert}[6]{ 
\path #1 ++(#4,0) coordinate (A); 
\path #2 ++(#4,0) coordinate (B); 
\draw [latex-latex] (A) -- (B) 
node [midway,fill=white] {#3}; 
\path #1 ++(#5,0) coordinate (C); 
\path #1 ++(#4+#6,0) coordinate (D); 
\draw (C) -- (D); 
\path #2 ++(#5,0) coordinate (C); 
\path #2 ++(#4+#6,0) coordinate (D); 
\draw (C) -- (D); 
 } 
% dibujar apoyos 
\apoyoemz{(0.000000,0.000000,0.000000)}{1} 
\apoyoemz{(4.000000,0.000000,0.000000)}{1} 
\apoyoemz{(8.000000,0.000000,0.000000)}{1} 
\apoyoemz{(12.000000,0.000000,0.000000)}{1} 
\apoyoemz{(0.000000,4.000000,0.000000)}{1} 
\apoyoemz{(4.000000,4.000000,0.000000)}{1} 
\apoyoemz{(8.000000,4.000000,0.000000)}{1} 
\apoyoemz{(12.000000,4.000000,0.000000)}{1} 
% dibujar elementos 
\tikzstyle{elem}=[double distance = 0.1cm,line cap=round]; 
\tikzstyle{elet}=[black,midway,above]; 
\tikzstyle{mat1}=[double=black!20]; 
\tikzstyle{mat2}=[double=red!20]; 
\tikzstyle{mat3}=[double=green!30]; 
\tikzstyle{mat4}=[double=blue!40]; 
\begin{scope} 
\draw[elem,mat1] (0.000000,0.000000,0.000000)--(0.000000,0.000000,4.000000) node[elet] {1}; 
\draw[elem,mat1] (4.000000,0.000000,0.000000)--(4.000000,0.000000,4.000000) node[elet] {2}; 
\draw[elem,mat1] (8.000000,0.000000,0.000000)--(8.000000,0.000000,4.000000) node[elet] {3}; 
\draw[elem,mat1] (12.000000,0.000000,0.000000)--(12.000000,0.000000,4.000000) node[elet] {4}; 
\draw[elem,mat1] (0.000000,4.000000,0.000000)--(0.000000,4.000000,4.000000) node[elet] {5}; 
\draw[elem,mat1] (4.000000,4.000000,0.000000)--(4.000000,4.000000,4.000000) node[elet] {6}; 
\draw[elem,mat1] (8.000000,4.000000,0.000000)--(8.000000,4.000000,4.000000) node[elet] {7}; 
\draw[elem,mat1] (12.000000,4.000000,0.000000)--(12.000000,4.000000,4.000000) node[elet] {8}; 
\draw[elem,mat1] (0.000000,0.000000,4.000000)--(0.000000,0.000000,8.000000) node[elet] {9}; 
\draw[elem,mat1] (4.000000,0.000000,4.000000)--(4.000000,0.000000,8.000000) node[elet] {10}; 
\draw[elem,mat1] (8.000000,0.000000,4.000000)--(8.000000,0.000000,8.000000) node[elet] {11}; 
\draw[elem,mat1] (12.000000,0.000000,4.000000)--(12.000000,0.000000,8.000000) node[elet] {12}; 
\draw[elem,mat1] (0.000000,4.000000,4.000000)--(0.000000,4.000000,8.000000) node[elet] {13}; 
\draw[elem,mat1] (4.000000,4.000000,4.000000)--(4.000000,4.000000,8.000000) node[elet] {14}; 
\draw[elem,mat1] (8.000000,4.000000,4.000000)--(8.000000,4.000000,8.000000) node[elet] {15}; 
\draw[elem,mat1] (12.000000,4.000000,4.000000)--(12.000000,4.000000,8.000000) node[elet] {16}; 
\draw[elem,mat2] (0.000000,0.000000,4.000000)--(4.000000,0.000000,4.000000) node[elet] {17}; 
\draw[elem,mat2] (4.000000,0.000000,4.000000)--(8.000000,0.000000,4.000000) node[elet] {18}; 
\draw[elem,mat2] (8.000000,0.000000,4.000000)--(12.000000,0.000000,4.000000) node[elet] {19}; 
\draw[elem,mat2] (0.000000,4.000000,4.000000)--(4.000000,4.000000,4.000000) node[elet] {20}; 
\draw[elem,mat2] (4.000000,4.000000,4.000000)--(8.000000,4.000000,4.000000) node[elet] {21}; 
\draw[elem,mat2] (8.000000,4.000000,4.000000)--(12.000000,4.000000,4.000000) node[elet] {22}; 
\draw[elem,mat1] (0.000000,0.000000,4.000000)--(0.000000,4.000000,4.000000) node[elet] {23}; 
\draw[elem,mat1] (4.000000,0.000000,4.000000)--(4.000000,4.000000,4.000000) node[elet] {24}; 
\draw[elem,mat1] (8.000000,0.000000,4.000000)--(8.000000,4.000000,4.000000) node[elet] {25}; 
\draw[elem,mat1] (12.000000,0.000000,4.000000)--(12.000000,4.000000,4.000000) node[elet] {26}; 
\draw[elem,mat2] (0.000000,0.000000,8.000000)--(4.000000,0.000000,8.000000) node[elet] {27}; 
\draw[elem,mat2] (4.000000,0.000000,8.000000)--(8.000000,0.000000,8.000000) node[elet] {28}; 
\draw[elem,mat2] (8.000000,0.000000,8.000000)--(12.000000,0.000000,8.000000) node[elet] {29}; 
\draw[elem,mat2] (0.000000,4.000000,8.000000)--(4.000000,4.000000,8.000000) node[elet] {30}; 
\draw[elem,mat2] (4.000000,4.000000,8.000000)--(8.000000,4.000000,8.000000) node[elet] {31}; 
\draw[elem,mat2] (8.000000,4.000000,8.000000)--(12.000000,4.000000,8.000000) node[elet] {32}; 
\draw[elem,mat1] (0.000000,0.000000,8.000000)--(0.000000,4.000000,8.000000) node[elet] {33}; 
\draw[elem,mat1] (4.000000,0.000000,8.000000)--(4.000000,4.000000,8.000000) node[elet] {34}; 
\draw[elem,mat1] (8.000000,0.000000,8.000000)--(8.000000,4.000000,8.000000) node[elet] {35}; 
\draw[elem,mat1] (12.000000,0.000000,8.000000)--(12.000000,4.000000,8.000000) node[elet] {36}; 
\end{scope} 
% dibujar nudos 
\tikzstyle{nudo}=[fill=black]; 
\tikzstyle{nudt}=[above=0.1cm]; 
\def \r{0.03cm}; 
\begin{scope} 
\draw[nudo] (0.000000,0.000000,0.000000) circle (\r) node[nudt] {1}; 
\draw[nudo] (4.000000,0.000000,0.000000) circle (\r) node[nudt] {2}; 
\draw[nudo] (8.000000,0.000000,0.000000) circle (\r) node[nudt] {3}; 
\draw[nudo] (12.000000,0.000000,0.000000) circle (\r) node[nudt] {4}; 
\draw[nudo] (0.000000,4.000000,0.000000) circle (\r) node[nudt] {5}; 
\draw[nudo] (4.000000,4.000000,0.000000) circle (\r) node[nudt] {6}; 
\draw[nudo] (8.000000,4.000000,0.000000) circle (\r) node[nudt] {7}; 
\draw[nudo] (12.000000,4.000000,0.000000) circle (\r) node[nudt] {8}; 
\draw[nudo] (0.000000,0.000000,4.000000) circle (\r) node[nudt] {9}; 
\draw[nudo] (4.000000,0.000000,4.000000) circle (\r) node[nudt] {10}; 
\draw[nudo] (8.000000,0.000000,4.000000) circle (\r) node[nudt] {11}; 
\draw[nudo] (12.000000,0.000000,4.000000) circle (\r) node[nudt] {12}; 
\draw[nudo] (0.000000,4.000000,4.000000) circle (\r) node[nudt] {13}; 
\draw[nudo] (4.000000,4.000000,4.000000) circle (\r) node[nudt] {14}; 
\draw[nudo] (8.000000,4.000000,4.000000) circle (\r) node[nudt] {15}; 
\draw[nudo] (12.000000,4.000000,4.000000) circle (\r) node[nudt] {16}; 
\draw[nudo] (0.000000,0.000000,8.000000) circle (\r) node[nudt] {17}; 
\draw[nudo] (4.000000,0.000000,8.000000) circle (\r) node[nudt] {18}; 
\draw[nudo] (8.000000,0.000000,8.000000) circle (\r) node[nudt] {19}; 
\draw[nudo] (12.000000,0.000000,8.000000) circle (\r) node[nudt] {20}; 
\draw[nudo] (0.000000,4.000000,8.000000) circle (\r) node[nudt] {21}; 
\draw[nudo] (4.000000,4.000000,8.000000) circle (\r) node[nudt] {22}; 
\draw[nudo] (8.000000,4.000000,8.000000) circle (\r) node[nudt] {23}; 
\draw[nudo] (12.000000,4.000000,8.000000) circle (\r) node[nudt] {24}; 
\end{scope} 
% dibujar ejes 
\draw (0,0,0)--(1,0,0) node[right] {$x$}; 
\draw (0,0,0)--(0,1,0) node[right] {$y$}; 
\draw (0,0,0)--(0,0,1) node[left] {$z$}; 
% dibujar cargas puntuales 
% dibujar cargas distribuidas uniformes en y-local 
0.000000,0.000000,4.000000)}{(4.000000,0.000000,4.000000)}{(-0.000000,0.000000,0.500000)}{20.000000}{midway,above} 
\cargadis{(4.000000,0.000000,4.000000)}{(8.000000,0.000000,4.000000)}{(-0.000000,0.000000,0.500000)}{20.000000}{midway,above} 
\cargadis{(8.000000,0.000000,4.000000)}{(12.000000,0.000000,4.000000)}{(-0.000000,0.000000,0.500000)}{20.000000}{midway,above} 
\cargadis{(0.000000,4.000000,4.000000)}{(4.000000,4.000000,4.000000)}{(-0.000000,0.000000,0.500000)}{20.000000}{midway,above} 
\cargadis{(4.000000,4.000000,4.000000)}{(8.000000,4.000000,4.000000)}{(-0.000000,0.000000,0.500000)}{20.000000}{midway,above} 
\cargadis{(8.000000,4.000000,4.000000)}{(12.000000,4.000000,4.000000)}{(-0.000000,0.000000,0.500000)}{20.000000}{midway,above} 
\cargadis{(0.000000,0.000000,8.000000)}{(4.000000,0.000000,8.000000)}{(-0.000000,0.000000,0.500000)}{20.000000}{midway,above} 
\cargadis{(4.000000,0.000000,8.000000)}{(8.000000,0.000000,8.000000)}{(-0.000000,0.000000,0.500000)}{20.000000}{midway,above} 
\cargadis{(8.000000,0.000000,8.000000)}{(12.000000,0.000000,8.000000)}{(-0.000000,0.000000,0.500000)}{20.000000}{midway,above} 
\cargadis{(0.000000,4.000000,8.000000)}{(4.000000,4.000000,8.000000)}{(-0.000000,0.000000,0.500000)}{20.000000}{midway,above} 
\cargadis{(4.000000,4.000000,8.000000)}{(8.000000,4.000000,8.000000)}{(-0.000000,0.000000,0.500000)}{20.000000}{midway,above} 
\cargadis{(8.000000,4.000000,8.000000)}{(12.000000,4.000000,8.000000)}{(-0.000000,0.000000,0.500000)}{20.000000}{midway,above} 
\end{tikzpicture} 
\end{center} 
\end{document} 
